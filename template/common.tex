% uncomment to polish language version
%\usepackage{polski}
\usepackage[utf8]{inputenc}
\usepackage[OT4]{fontenc}
\usepackage{times}
%\usepackage[T1]{fontenc}
\usepackage{listings,bera}
\usepackage{graphicx}
\usepackage[absolute,overlay]{textpos}
\usepackage{xcolor}
\usepackage{multicol}
\usepackage{multirow}
\usepackage{etoolbox}
\usepackage{tikz}
\usepackage{pifont}
\usepackage[normalem]{ulem}
% \usepackage{url}
\usepackage{hyperref}
\usepackage{minted}

% uncomment this out to enable links highlighting (note: will affect redirections everywhere!):
%\hypersetup{colorlinks=true}

% disables the navigation symbol on the bottom of the screen
\setbeamertemplate{navigation symbols}{}
% disables slide numbers on the bottom of the screen
\setbeamertemplate{footline}{}

%\usecolortheme[rgb={0.1,0.6,0.3}]{structure}

\title{Hello World!}
\subtitle{C++ is Fun :-D}

\author[BaSz]{Bartek 'BaSz' Szurgot}
%\institute{my@e.mail}
\institute{\url{https://www.baszerr.eu}}

% comment this out to set date to current one, in a given language
\date{2016-06-06}

% comment this out to disable outline slides, between sections
\AtBeginSection[]
{
\begin{frame}<beamer>
%\frametitle{Outline for section \thesection}
\tableofcontents[currentsection]
% \begin{center}
% \textbf{
% \fontsize{40}{50}\selectfont
% \insertsection
% }
% \end{center}
\end{frame}
}



% insImgFr{frame-range}{x}{y}{s}{img}
\newcommand{\insImgFr}[5]
{
  \begin{textblock*}{5cm}(#2\textwidth,#3\textwidth)
    \includegraphics<#1>[scale=#4]{#5}
  \end{textblock*}
}


% insImg{x}{y}{s}{img}
\newcommand{\insImg}[4]
{
  \insImgFr{0-}{#1}{#2}{#3}{#4}
}


% insImgCenter{s}{img}
\newcommand{\insImgCenter}[2]
{
  \begin{center}
    \includegraphics[scale=#1]{#2}
  \end{center}
}


% \subSlide{name}{content}
% pass an empty 'name' to create unnamed sub-slide
\newcommand{\subSlide}[2]
{
\usemintedstyle{xcode}
% \usemintedstyle{tango}
% \usemintedstyle{manni}
% \usemintedstyle{perldoc}
% \usemintedstyle{autumn}
% \usemintedstyle{rainbow_dash}
\begin{frame}
\frametitle{#1}
#2
\end{frame}
}


% \slide{name}{content}
% pass an empty 'name' to create unnamed slide
\newcommand{\slide}[2]
{
\subsection*{section#1}
\subSlide{#1}{#2}
}


% \transparent{transparency/0-100/=30}{text}
\newcommand{\transparent}[2]
{\color{fg!#1}#2}


% \insSrc{lang}{fileName}
\newcommand{\insSrc}[2]
{
\inputminted[linenos,
             breaklines=true]
             {#1}{#2}
}

% \insCppSrc{fileName}
\newcommand{\insCppSrc}[1]
{
\insSrc{cpp}{cpp/#1.hpp}
}

% \insCppSrcHl{line(s)}{fileName}
\newcommand{\insCppSrcHl}[2]
{
\inputminted[linenos,
             breaklines=true,
             highlightcolor=lightgray,
             highlightlines=#1]
             {cpp}{cpp/#2.hpp}
}


% \insPlaceholder{what}
\newcommand{\insPlaceholder}[1]
{
\transparent{0}{#1}
}


% \sourceRefUrlShifted{h_offset}{url}
\newcommand{\sourceRefUrlShifted}[2]
{
  \begin{center}
  \color{gray}
  \fontsize{4}{5}
  \selectfont
    \textit{\mbox{\url{#2}\hspace{-#1}}}
  \end{center}
}


% \sourceRefUrl{url}
\newcommand{\sourceRefUrl}[1]
{
  \sourceRefUrlShifted{0.0001em}{#1}
}
